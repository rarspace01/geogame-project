\chapter*{Abstract}
\label{ch0:abstract}

\begin{tabbing}
Verfasser:	 \hspace{20mm} \= Denis Hamann\\
Thema:	  \> Konzeption und Umsetzung eines ortsbezogenen\\
		\> Gamification-Ansatzes für regionale Dienstleister\\
Keywords: \> Pervasive Games, automatic Relocation, magic circle,\\
		\> Game Framework, OpenStreetMap
\end{tabbing}
\vspace{1cm}
In der vorliegenden Arbeit soll ein Framework entwickelt werden, welches einen Spielleiter mit Hilfe von OpenStreetMap-Daten und Gamification bei der Durchführung eines ortsbezogenen Spiels unterstützt. Hierbei soll eine Integration von regionalen Dienstleistern in das Framework erfolgen.
Zunächst wird auf die Probleme von ortsbezogenen Spielen eingegangen.
Der Fokus wird insbesondere auf die Relokalisierbarkeit von Spielfeldern in Pervasive Games und der Verwendung von OSM-Daten gelegt. Der State of the Art der Literatur wird für mögliche Lösungen einer Relokalisierung herangezogen. Dabei wird ein eigener Ansatz mittels OSM-Daten vorgestellt. In diesem Zusammenhang wird auf die Transformation der Daten von OSM zu Spielelementen eingegangen.
Es werden zudem Ansätze diskutiert, wie regionale Dienstleister unter Zuhilfenahme von neuster Technologie in das Spiel eingebunden werden können.
Im Anschluss wird eine entsprechender Entwurf für die ausgewählte Lösung präsentiert und dieser entsprechend umgesetzt. Nach der Umsetzung erfolgt die Evaluation des Frameworks sowie der Spielfelder. Hierbei wird festgehalten, dass die Qualität der OSM-Daten für die Generierung der Spielfelder eine ausreichende Güte aufweist. Darüber hinaus werden Methoden vorgestellt, wie Spielelemente auf gewichteten Netzwerken bewertet werden können. Zum Schluss werden weitere Optimierungspotentiale aufgezeigt und der Beitrag der Arbeit zur Lösung der Fragestellung diskutiert.