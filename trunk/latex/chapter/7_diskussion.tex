\chapter{Diskussion}
\label{sec:S7_Diskussion}

\section*{Einordnung der Ergebnisse}

Nach der Evaluation des Frameworks und des Beispielspiels kann festgehalten werden, dass die zuvor in der Problemstellung geschilderten Anforderungen anhand des beschriebenen Frameworks angegangen werden können. Um der Echtzeitanforderung gerecht zu werden, wurde die Auswahl der Spielelemente anhand von OSM-Daten auf eine zweistufiges Verfahren ausgelegt. Der ressourcenaufwendigere Part wurde ausgegliedert und vor das Pervasive Game gestellt. Hierzu wurde ein Evaluationstool entwickelt welches anhand einer gegebenen Position und eines Key-Value Paars die Spielfläche automatisch bewertet und für einen Vergleich mit anderen Spielfeldern herangezogen werden kann.
Die Auswahl wiederum kann im Framework entsprechend festgehalten werden. Es wurde festgestellt, dass gewisse Stellen weiter einer Optimierung bedürfen um noch bessere Ergebnisse zu erzielen. Hierunter fällt die Bewertung der Entfernungsmatrix welche für die einzelnen Spielelemente der Transformation erzeugt wird. Erste Ansätze und Ergebnisse wurden entsprechend geschildert, sowie weitere Vorgehensweisen präsentiert.
Es wurde eine funktionierende Transformation für die Umwandlung von OSM-Elementen zu Spielelementen entworfen und entsprechend umgesetzt.


\section*{Relokalisierbarkeit geobasierter Gamification-Ansätze}

\section*{Ausblick}

-Evaluation Spielfelder ausbauen
-Framework untersuchen/optimieren auf Hochskalierbarkeit
-reales Spielfeedback, iterativ