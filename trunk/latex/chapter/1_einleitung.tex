\chapter{Einleitung: Ortsbezogene Gamification}
\label{ch1:S1_Einleitung}

Im Zuge der Digitalisierung steht der regionale Einzelhandel vor der Herausforderung für seine Kunden weiterhin interessant zu sein, bei gleichzeitig steigender Konkurrenz durch das Internet. Aktuelle Zahlen des Statistischen Bundesamtes (http://deutsche-wirtschafts-nachrichten.de/2012/11/30/einzelhandel-mit-staerkstem-umsatzeinbruch-seit-vier-jahren-2/) und der GfK belegen einen stagnierenden bzw. teilweise einen rückläufigen Markt.
Hierbei stellt sich die Frage in welcher Art und Weise die regionalen Händler bestehende Kunden binden können aber auch neue Kunden mit deren Angebot vertraut gemacht werden, welche dieses noch nicht im Detail kennen.
Klassische Marketing Ansätze sind verbreitet und werden entsprechend genutzt.

Ein Ansatz der in den letzten Jahren immer mehr an Bedeutung gewonnen hat, stellt die Gamification dar. Durch diese wird versucht eine extrinsische Motivation für bestimmte Handlungen zu erzeugen. In diesem Zusammenhang soll durch Gamification eine Kundenbindung und -neugewinnung erzielt werden.

Um ein interessantes Spielkonzept dem Kunden bieten zu können wird hierbei auf Geogames zurück gegriffen.
Ein solches nutzt die physkalische Fortbewegung der Spieler als Interaktion mit dem Spiel.
Durch diesen Modus kann erreicht werden, dass die Spieler geografisch mit entsprechenden Orten interagieren und im konkreten Fall regionale Anbieter aufsuchen.

Der Hauptaugenmerk dieser Arbeit soll sich vor allem auf die konkrete Erstellung der Spielfelder eines Geogames beziehen. Hierzu soll zunächst untersucht werden, inwiefern bestehende öffentliche Datenbanken wie z.B. Openstreetmaps (OSM) genutzt werden können. Auf der anderen Seite müssen diese Daten entsprechend aufbereitet werden und dem Spiel zur Verfügung gestellt werden.

Daher fokussiert sich die Fragestellung der Arbeit wie folgt:
\begin{itemize}
  \item Wie können freie Geobasisdaten zur Konfiguration von Spielfeldern genutzt werden?
\end{itemize}

Die konkrete Umsetzung soll anhand eines Beispielspiels erfolgen.
Hierfür wird ein entsprechender Entwurf erstellt, der dies ermöglichen soll.

% Konzeption und Umsetzung eines ortsbezogenen Gamification-Ansatzes für regionale Dienstleister