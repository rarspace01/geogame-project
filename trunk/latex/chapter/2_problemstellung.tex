\chapter{Problemstellunug}
\label{ch2:Problemstellunug}

\subsection*{Möglichkeiten ortsbezogener Gamification}

Ziel der Arbeit ist es, ein Framework zu erstellen, welches Gamification Elemente verwendet um eine Plattform für Spiele zu bieten. Diese Spiele dienen idealerweise dazu, dass durch den Effekt der Gamification neue oder bestehende Kunden an die lokalen Dienstleister gebunden werden. Es soll dabei untersucht werden, welche Möglichkeiten und Vorgehensweisen es in der Literatur gibt und wie diese sich in ein Framework integrieren lassen.
Durch die Integration wird sich versprochen, dass die Kunden dadurch eine extrinsische Motivation erfahren. Diese soll die Kunden zu bestimmten Interaktionen mit den einzelnen Geschäften animieren.
Dabei soll aufgezeigt werden, wie eine Interaktion des Kunden vor Ort beim Dienstleister aussehen kann. Eine konkrete Evaluation der Effekte durch das Framework ist nicht Bestandteil der Arbeit.

\subsection*{Location-based Games als Mittel für Gamification}

Im Zuge der Arbeit soll mit Hilfe eines Beispiel Spiels die Brauchbarkeit des zuvor angekündigten Frameworks untersucht werden.
Das Spiel selbst soll ein Location-based Game darstellen. Ortsbezogene Spiele haben gewisse Anforderungen, die es zu erfüllen gibt. Hierzu 
ist es notwendig die Anforderungen beim Entwurf und bei der Umsetzung eines Frameworks zu berücksichtigen.
Durch die Verwendung der Gamification Elemente, die durch das Framework zur Verfügung gestellt werden, soll das Spiel aufgewertet werden.
Dadurch ist es notwendig, gewisse Entscheidungen bezüglich des Spielinhalts als auch der Spielmechanik zu machen.
Diese Maßnahmen sollen aufgezeigt und erläutert werden.

\subsection*{Anforderungen an ein Geogameframework}

Durch die Verwendung eines ortsbezogenen Spieles es notwendig, dass das Framework gewissen Anforderungen gerecht wird.
Diese Anforderungen müssen für den Entwurf des Frameworks identifiziert und für die Umsetzung beachtet werden.
Eins der Ziele ist es dabei dem Spielleiter möglichst viel Arbeit abzunehmen und gleichzeitig Konfigurationsmöglichkeiten zur Verfügung zu stellen.

\subsection*{Relokalisierbarkeit von ortsbezogenen Spielen}

Ein wichtiger Aspekt bei ortsbezogenen Spielen stellt der georeferenzierte Spielinhalt dar. Dieser beinhaltet georeferenzierte Spielelemente mit denen der Spieler über das Spielfeld interagiert.
Die Herausforderung ist es, dass ein Spiel nicht nur in einem fest definierten Bereich funktioniert. Das Framework soll es ermöglichen, dass das Spiel auch außerhalb geprüfter Orte spielbar ist.
Die Lösungsansätze sollen mit Hilfe der Literatur untersucht und gegebenenfalls die Ansätze für einen Lösungsversuch aufgegriffen werden.
Hierbei muss sicher gestellt werden, dass ein Kompromiss zwischen Komplexität und der Laufzeit des Verfahrens gefunden wird. Letzteres beeinflusst die Spielbarkeit.
Bei der Verwendung eines Pervasive Games ist es daher notwendig, dass je nach Spielmechanik, die Erzeugung von Spielfeldern in Echtzeit erfolgen muss.
Diese und andere Aspekte der Relokalisierung sollen identifiziert und nach einer Bewertung im Framework umgesetzt werden.

\subsection*{Freie Geobasisdaten und Möglichkeiten der kommerziellen Nutzung}

Das Ziel der Arbeit ist die Erstellung des Spielinhalts mit Hilfe von OSM-Daten zu unterstützen.
Bei der Verwendung von freien Geobasisdaten stellen sich diverse Fragen. Hierunter fallen die Qualität der Daten, als auch die Frage nach der Lizenz.
Freie Geobasisdaten stehen in Konkurrenz zu kommerziellen Daten. Die Verwendbarkeit von Ersteren im Hinblick auf die Anforderungen für ein Framework soll bewertet werden.
Der Einsatz von OSM soll daher auf Basis der Anforderungen und unter Zuhilfenahme der Literatur bewertet werden. Im Anschluss an die technische Umsetzung des Frameworks soll untersucht werden, inwiefern die Daten sich für den Einsatz in einem Frameworks eignen.
