\chapter{Lösungsansatz}
\label{sec:S4_Lösungsansatz}

\section{Mögliche Lösungen}

\subsection*{Gamification}

Im Hinblick auf die Problemstellung in Kapitel \ref{ch2:Problemstellunug}, sowie der in der Literatur beschriebenen Ansätze in Kapitel \ref{ch3:s:Gamification}, ergeben sich mehrere Möglichkeiten für eine Lösung. Ziel ist es die Kunden zum einen zu motivieren neue Geschäfte aufzusuchen und auf der anderen Seite eine Bindung zu bestehenden Geschäften zu erreichen. Da es sich hierbei um um einen Prozess handelt, der für den Kunden im ersten Schritt nur einen geringen Benefit darstellt, ist dieser ideal für Gamification geeignet \cite{Leigh.2012}. Ziel ist es somit die Besuche des Spielers durch eine Gamification interessant zu machen. Dies kann durch die verschiedenen Aspekte wie das Vergeben von Punkten für einen Einkauf (siehe Miles \& More \cite{Wagner.2005}, Payback \cite{Roesl.2005}) erreicht werden. Allerdings sind reine, einfache Punkteprogramme weit verbreitet und deren Nutzen umstritten \cite{Schmitt.2001}.
Daher ist es notwendig, das komplette Spektrum der Gamification zu betrachten.
Über das reine Punkte System hinaus, manifestiert sich in der Literatur die Grundmenge "Points, Badges und Leaderboards".
Konkret bedeutet dies, neben der einfachen Ansammlung von Punkten, gibt es darüber hinaus Badges/Auszeichnungen welche das Engagement des einzelnen Nutzers widerspiegeln. Ein weiterer Aspekt sind die sogenannten Leaderboards bzw. Bestenlisten. Über diese wird wird ein Ansporn unter den jeweiligen Spielern erzeugt die anderen Spieler zu übertrumpfen. Vergleicht man diese Ansätze mit denen der typischen Spieler Profile nach \citep{Bartle.2004} so erkennt man, dass nicht alle dieser mit den Standard Elementen bedient werden.

\begin{itemize}
\item Achievers
\item Socializers
\item Explorers
\item Killers
\end{itemize}

Die standardtypischen Elemente der Gamification bedienen vorzugsweise Achievers und Killers.
Die Frage stellt sich hierbei, wie Socializers und Explorers bedient werden können.


%
\subsection*{Anforderungen an ein Geogameframework}

\subsection*{Relokalisierbarkeit von ortsbezogenen Spielen}

\subsection*{Freie Geobasisdaten und Möglichkeiten der kommerziellen Nutzung}

\section{Gewählter Lösungsansatz}


