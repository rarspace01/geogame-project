\section{Glossar}

%\newpage{Abkrzugsverzeichnis}
%\addcontentsline{toc}{chapter}{Abkrzungsverzeichnis}
%%\addcontentsline{toc}{chapter}{\protect{Glossar}}
\begin{acronym}[abkuerzungen2]
\acro{API}{Die API stellt eine dokumentierte Software-Schnittstelle dar, die von anderen Programmen aus genutzt werden kann.}
\acro{Basismaschine}{Eine Basismaschine stellt Datenobjekte und Operatoren bereit, auf deren Grundlage die Datenobjekte und Operatoren der Nutzermaschine realisiert werden.}
\acro{CLI}{Command Line Interface - Kommandozeile. Die Kommandozeile ist ein Eingabebereich für die Steuerung einer Software, die typischerweise im Textmodus abläuft.}
\acro{DNS}{Ermöglicht es Klarnamen in numerische IP Adressen (z.B. google-public-dns-a.google.com in 8.8.8.8 umzuwandeln).}
\acro{GUI}{Hierbei handelt es sich um die grafische Benutzeroberfläche.}
\acro{IP}{Ein Protokoll das für die Vermittlung von Daten dient.}
\acro{ISO}{Eine internationale Vereinigung von Normungsorganisationen.}
\acro{JDBC}{Hierbei handelt es sich um eine Datenbankschnittstelle für Java.}
\acro{Klasse}{Im Kontext der Programmierung handelt es sich hierbei um einen abgegrenzten Bereich (ein sogenanntes "`Objekt"') mit bestimmten Attributen und Methoden.}
\acro{LDAP}{Ein Verzeichnisdienst um Abfragen und Modifikationen von Informationen zu erlauben.}
\acro{ODBC}{Hierbei handelt es sich um eine Datenbankschnittstelle von Microsoft.}
\acro{RFC}{RFCs sind eine Reihe von technischen und organisatorischen Dokumenten zum Internet, die sie zu einem Standard entwickelt haben.}
\acro{Salt}{Ein Salt (zu Deutsch "`Salz"') wird bei der Erstellung einer Prüfsumme zu einem Passwort beigegeben, um zu gleichen Passwörtern unterschiedliche Prüfsummen zu erhalten. Ohne ein Salt wäre anhand der Prüfsumme das Passwort zu erkennen, wenn einmal bekannt ist, welche Prüfsumme zu welchem Passwort gehört.}
\acro{SHA-256}{SHA steht für "`Secure Hash Algorithm"' wobei die angefügte Zahl die Länge der generierten Prüfsumme ("`Hash"') in Bit angibt.}
\acro{Shell}{Eingabe-Schnittstelle zwischen Computer und Benutzer}
\acro{SQL}{Eine deskriptive Abfragesprache von Datenbanken.}
\acro{TCP}{Ein verbindungsorientiertes Protokoll, um Daten im Netzwerk zu transportieren.}
\acro{UDP}{Ein verbindungsloses Protokoll, um Daten im Netzwerk zu transportieren.}
\end{acronym}
