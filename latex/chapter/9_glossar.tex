\chapter{Glossar}

%\newpage{Abkrzugsverzeichnis}
%\addcontentsline{toc}{chapter}{Abkrzungsverzeichnis}
%%\addcontentsline{toc}{chapter}{\protect{Glossar}}
\begin{acronym}[abkuerzungen2]
\acro{API}{Die API stellt eine dokumentierte Software-Schnittstelle dar, die von anderen Programmen aus genutzt werden kann.}
\acro{CLI}{Command Line Interface - Kommandozeile. Die Kommandozeile ist ein Eingabebereich für die Steuerung einer Software, die typischerweise im Textmodus abläuft.}
\acro{DB Lounge}{Lounge der Deutschen Bahn für Status Kunden.}
\acro{DNS}{Ermöglicht es Klarnamen in numerische IP Adressen (z.\,B. google-public-dns-a.google.com in 8.8.8.8 umzuwandeln).}
\acro{DSL}{domain-specific language - Eine domänenspezifische Sprache}
\acro{GUI}{Hierbei handelt es sich um die grafische Benutzeroberfläche.}
\acro{OQL}{Overpass Query Language}
\acro{OSM}{OpenStreetMap}
\acro{RFC}{RFCs sind eine Reihe von technischen und organisatorischen Dokumenten zum Internet, die sie zu einem Standard entwickelt haben.}
\acro{POI}{Pointn of Interest. Sehenswürdigkeit}
\acro{PBL}{Points, Badgets, Leaderboards}
\acro{Shell}{Eingabe-Schnittstelle zwischen Computer und Benutzer}
\acro{SQL}{Eine deskriptive Abfragesprache von Datenbanken.}
\end{acronym}
