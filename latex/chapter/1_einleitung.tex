\chapter{Einleitung: Ortsbezogene Gamification}
\label{ch1:Einleitung}

Im Zuge der Digitalisierung steht der regionale Einzelhandel vor der Herausforderung für seine Kunden weiterhin interessant zu sein, bei gleichzeitig steigender Konkurrenz durch das Internet. Aktuelle Zahlen des Statistischen Bundesamtes \citep{DWN.2012} und der GfK belegen einen stagnierenden bzw. teilweise einen rückläufigen Markt.
Hierbei stellt sich die Frage in welcher Art und Weise die regionalen Händler bestehende Kunden binden können neue Kunden mit deren Angebot vertraut gemacht werden, welche dieses noch nicht im Detail kennen.
Klassische Marketing Ansätze sind verbreitet und werden entsprechend genutzt.
Ein Ansatz der in den letzten Jahren mehr an Bedeutung gewonnen hat, stellt die Gamification dar. Durch diese wird versucht eine extrinsische Motivation für bestimmte Handlungen zu erzeugen. soll durch Gamification eine Kundenbindung und -neugewinnung erzielt werden.

Um ein interessantes Spielkonzept dem Kunden bieten zu können wird hierbei auf ortsbezogenen Spiele zurück gegriffen.
Ein Solches nutzt die physischen Fortbewegung der Spieler als Interaktion mit dem Spiel.
Durch diesen Modus kann erreicht werden, dass die Spieler geografisch mit entsprechenden Orten interagieren und im konkreten Fall regionale Anbieter aufsuchen.

Der Hauptaugenmerk dieser Arbeit soll sich vor allem auf die konkrete Erstellung der Spielfelder eines Geogames beziehen. Hierzu soll zunächst untersucht werden, inwiefern bestehende öffentliche Datenbanken konkret am Beispiel von OpenStreetMap (OSM) genutzt werden können. Auf der anderen Seite müssen diese Daten entsprechend aufbereitet werden und dem Spiel zur Verfügung gestellt werden.

Daher fokussiert sich die Fragestellung der Arbeit wie folgt:
\begin{quote}
\item Wie können Spielfelder anhand von OSM-Daten mithilfe eines Frameworks erstellt werden?
\end{quote}
\vspace{1cm}
%\begin{itemize}
% \item Wie können freie Geobasisdaten zur Konfiguration von Spielfeldern genutzt werden?
%\end{itemize}

Es wird daher untersucht, in welcher Art und Weise die Daten vorliegen und entsprechend transformiert werden müssen.
Darüber hinaus muss wird untersucht, wie eine Bewertung der Spielelemente stattfinden kann um möglichst gute Spielfelder zu erstellen.
Ziel ist es einem Spielleiter durch ein Framework die Möglichkeit zu geben, den Aufwand für das Staging von Spiels zu reduzieren. Durch den Einsatzes des Frameworks sollen gleichzeitig lokale Gewerbe in das Spiel eingebunden werden können. Dadurch sollen Spieler auf Geschäfte oder andere Dienstleister aufmerksam gemacht werden.
Die konkrete Umsetzung soll anhand eines Beispielspiels erfolgen.
Hierfür wird ein entsprechender Entwurf erstellt, der dies ermöglichen soll.
Die Arbeit behandelt nicht die Evaluation bezüglich der Umsatzsteigerung von regionalen Dienstleistern. Vielmehr soll das entsprechende Framework die Möglichkeit dazu bieten.

Die Vorgehensweise der Arbeit beginnt mit einer Problembeschreibung. Anhand dieser wird der aktuelle Forschungsstand in der Literatur untersucht werden. Im Anschluss werden Lösungsansätze vorgestellt und entsprechende Entscheidungen erläutert. Nach der Lösungsauswahl wird der konkrete Entwurf umgesetzt. Hierbei wird auf die technischen Details der Lösung eingegangen. Im Anschluss erfolgt die Evaluation der vorgestellten Lösung. Hierbei wird begutachtet inwiefern der vorgestellte Lösungsansatz die beschriebene Problemstellung lösen kann und einen Beitrag zur Forschung leistet. Im Anschluss werden die Ergebnisse diskutiert und entsprechende weitere Vorgehen identifiziert, welche im Ausblick behandelt werden.

% Konzeption und Umsetzung eines ortsbezogenen Gamification-Ansatzes für regionale Dienstleister