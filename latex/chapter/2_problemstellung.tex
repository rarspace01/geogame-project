\chapter{Problemstellunug}
\label{sec:S2_Problemstellunug}

\section{Möglichkeiten ortsbezogener Gamification}

Gamification als Methode in Kapitel \ref{subsec:S3_Gamification} vorgestellt.
Kann dazu verwendet werden um entsprechende Handlungne extrinsich vom Nutzer zu motivieren.
Ziel ist es Gamification Elemente entsprechend so einzusetzten, dass der Nutzer dazu motiviert wird die entsprechenden Gewerbe zu besuchen und dabei potentiell Umsatz beim Unternehmer zu generieren.
 

\section{Geogames als Mittel für Gamification}

\section{Anforderungen an ein Geogameframework}

\section{Relokalisierbarkeit von ortsbezogenen Spielen}

\section{Freie Geobasisdaten und Möglichkeiten der kommerziellen Nutzung}