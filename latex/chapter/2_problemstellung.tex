\chapter{Problemstellunug}
\label{ch2:Problemstellunug}

\subsection*{Möglichkeiten ortsbezogener Gamification}

Ziel der Thesis ist es, durch ortsbezogene Gamification neue oder bestehende Kunden an die lokalen Dienstleister zu binden.
Hierzu muss untersucht werden, welche Möglichkeiten und Vorgehensweisen zu empfehlen sind.
Hierbei sollen die Kunden durch extrinsische Motivation zu bestimmten Interaktionen mit den einzelnen Geschäften animieren werden.
Durch die zusätzlich Interaktion soll entsprechend potentiell mehr Umsatz gemacht werden.

\subsection*{Location based Games als Mittel für Gamification}

Im Zug der Arbeit gilt es zu überprüfen wie Location based Games am besten eingesetzt werden können und wie diese gestaltet werden müssen um den Ansprüchen der Gamification zu genügen. Hierbei sollen gleichzeitig die erwünschten Effekte (konkret mehr Umsatz für den Händler) eintreten. Um diesen gerecht zu werden muss das Design des Spieles bezüglich Content aber auch Spielmechanik entsprechend ausgerichtet werden.

\subsection*{Anforderungen an ein Geogameframework}

Die Anforderung an ein entsprechendes Framework muss untersucht werden. Es muss festgestellt werden, wie einem Entwickler bzw. einem Spielveranstalter möglichst viel Arbeit abgenommen werden kann und gleichzeitig trotzdem entsprechende Konfigurationsmöglichkeiten zur Verfügung gestellt werden.

\subsection*{Relokalisierbarkeit von ortsbezogenen Spielen}

Ein wichtiger Aspekt bei ortsbezogenen Spielen stellt der Game Content bzw. der Geogamecontent dar. Hierbei handelt es sich um georeferenzierte Spielelemente mit denen der Spiele über das Spielfeld interagiert. Die Herausforderung ist es, dass ein Spielfesd nicht nur in einem fest definierten Bereich, z.B. in einer speziellen Stadt gespielt werden kann, sondern auch an anderen Orten. Hierfür gibt es unterschiedliche Ansätze die es zu beleuchten gibt und zu entscheiden ist, welcher dieser am sinnvollsten ist.

\subsection*{Freie Geobasisdaten und Möglichkeiten der kommerziellen Nutzung}

Es gibt viele kommerzielle und nicht kommerzielle Geodaten die heutzutage bezogen werden können. Es gilt zu untersuchen, welche ob öffentliche bzw. freie Geodaten heutzutage verwenden werden können und wie diese in Verbindung mit einem kommerziellen Projekt genutzt werden können.