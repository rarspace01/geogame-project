\section{Forschungsstand}
\label{sec:S3_Forschungsstand}

\subsection{Gamification}

Der Begriff Gamification geht auf Nick Pelling 2002 zurück (Marczewski, Andrzej (April 2012). "Forward". Gamification: A Simple Introduction (1st ed.). p. 46., http://nanodome.wordpress.com/2011/08/09/the-short-prehistory-of-gamification/)

Es gibt verschiedene Definitionen:

-Definition von Gamification:

Oxford:
"the application of typical elements of game playing (e.g. point scoring, competition with others, rules of play) to other areas of activity, typically as an online marketing technique to encourage engagement with a product or service:
gamification is exciting because it promises to make the hard stuff in life fun"

"The process of game-thinking and game mechanics to engage users and solve problems"
(Zichermann, Cunningham (2011, S. XIV)

"Gamification is the use of game design elements in non-game context"
(Deterding, Dixon et al. (2011, S2.)

Kapp "Gamification is using game based mechanics, aesthetics and game thinking to engange people, motivate action, promote learning, and solve problems"
Kapp, 2012, S.10


"Title" - (SEBASTIAN, DAN, RILLA and LENNART 2011)

ODOBSAIC et. al Gamificatoin of geographic data collection (2013):

\begin{figure}[H]
\begin{center}
\includegraphics[width=80mm]{images/ch3_img01_LBG_SN_etc.png}
\caption{Einordnung Definition}
\label{img:ch3_img01_LBG_SN_etc}
\end{center}
\end{figure}

Bei allen:
Nutzung von Spieldenken und Spielmechaniken

Bei Zichermann \& Kapp:
Motivatoin \& Problemlösung

aber auch "Empowerement":
Games have a strong ability of imparting a sense of agency to the players, making them feel
empowered and giving them the impression that their decisions are meaningful and will
have an impact. A sense of agency refers to the subjective awareness that one is initiating,
executing and controlling one’s own volitional actions in the world (JEANNEROD 2003).

Sonst:
Spielferner Context
Lernförderung, Anregung zum Handeln

-Es werden verschiedne Elemente genutzt um Spieler zu spielfremden Verhalten (hier: Besuch von Geschäften) zu motivieren


Zichermann \& Cunningham (2011):
Mensch hat hang zum Spielen.

SAPS (do not give something for free, but something for status etc.)

\begin{itemize}
      \item Status (Badges, Levels/Leaderboards)
      \item Access (early Access)
      \item Power (give power, e.g. modicum control over other players
      \item Stuff (give a reward, try to prevent that the price gets known)
\end{itemize}

The house always wins

"flow"
Game designers aim to
keep a player in a constant state of flow. Flow indicates the state in which a player is between
anxiety and boredom and meeting their own motivational level in that experience
(CSIKSZENTMIHALYI 1991).

Game Mechanics:
\begin{itemize}
      \item Points
      \item Levels
      \item Leaderboard
      \item Badges
      \item Challenges
      \item Onboarding
      \item engagement loops
\end{itemize}

\subsection{Geogames}

Game/Spiel:

According to  Katie  Salen  and  Eric  Zimmerman  (2004):
"Magic circle"

Mobilegames
-games played on mobile devices

Locationbased Games
games played in a location context

"Extending Cyberspace:Location Based Games Using Cellular Phones"

Pervasive Games

bluring the magic circle in "Montola, Markus. "Exploring the edge of the magic circle: Defining pervasive games." Proceedings of DAC. 2005."

"A pervasive game is agame that has one or more salient features that expand the 
contractual magie circle of play spatially, temporally, or socially. " @ Motola et al 2009

" This   definition   has   been   discussed   earlier  in   Montola   (2005)   and   in   Montola,   Waern,   and 
Nieuwdorp  (2006).  Staffan  Björk  (2007)  has  also  published  an  alternate version,  where  ambiguity 
of  interaction or  interface is  included  as  a  fourth  central  defining ceriterion." 

Can You See Me Now (CYSMN) (Flintham et al., 2003a)
GeoTicTacToe and CityPoker (see Schlieder et al. 2005a, Schlieder 2005b)
Human Pacman, realised by Choek et al. (2004)

Unterscheidung zwischen LBG, AR MR etc:

Concepts and Technologies for Pervasive Games

\subsection{Relokalisierungsansätze}

Literatur von Kinf, sonst keine in diesem Umfeld.
Relocalisierungsansätze werden zwar angesprochen, z.:B in 
Exploring the Edge of the Magic Circle: 
Defining Pervasive Games 
Markus Montola 

es wird auf eventuelle Implikationen hingewiesen aber nciht im detail weiter verfolgt.

\subsection{Verwendung offener Geodaten}